\documentclass[OPS,authoryear,lsstdraft,toc]{lsstdoc}
\input{meta}

% Package imports go here.
\usepackage{graphicx} 

% Local commands go here.

%If you want glossaries
%\input{aglossary.tex}
%\makeglossaries

\title{Rubin Science Performance Metrics}

% Optional subtitle
% \setDocSubtitle{A subtitle}

\author{%
Leanne Guy, ....,  and the DM System Science Team
}

\setDocRef{RTN-038}
\setDocUpstreamLocation{\url{https://github.com/lsst/rtn-038}}

\date{\vcsDate}

% Optional: name of the document's curator
% \setDocCurator{The Curator of this Document}

\setDocAbstract{%
The document provides the detailed mathematical definitions for all Rubin/LSST science performance metrics. 
It provides quantitative definitions for all the science performance metrics specified in the LSST Science Requirements document and derived documents as well as additional metrics defined as part of system development. 
}

% Change history defined here.
% Order: oldest first.
% Fields: VERSION, DATE, DESCRIPTION, OWNER NAME.
% See LPM-51 for version number policy.
\setDocChangeRecord{%
  \addtohist{1}{2023-07-10}{First draft}{Leanne Guy}
}

\graphicspath{{figures/}} %Setting the graphicspath

\begin{document}

% Create the title page.
\maketitle
% Frequently for a technote we do not want a title page  uncomment this to remove the title page and changelog.
% use \mkshorttitle to remove the extra pages

% ADD CONTENT HERE
\section{Introduction}


\section{Science Requirements Document Metrics}

The LSST science requirements document and associated derived documents define a set of \textit{Science Performance Metrics} that are used to quantify the performance of the LSST. 

\subsection{Image Quality Metrics}

 \subsection{Photometric Metrics}
 
 \subsection{Astrometric Metrics} 
\section{Additional Science Performance Metrics}

\appendix
 \section{Metric definition template}
\label{app:metric-template}

The following template should be used to define all metrics in this document

\begin{table}[htp]
\caption{default}
\begin{center}
\begin{tabular}{|c|c|}
\hline
Details | to come \\\hline

\hline\hline
\end{tabular}
\end{center}
\label{default}
\end{table}%


% Include all the relevant bib files.
% https://lsst-texmf.lsst.io/lsstdoc.html#bibliographies
\section{References} \label{sec:bib}
\renewcommand{\refname}{} % Suppress default Bibliography section
\bibliography{local,lsst,lsst-dm,refs_ads,refs,books}

% Make sure lsst-texmf/bin/generateAcronyms.py is in your path
\section{Acronyms} \label{sec:acronyms}
\input{acronyms.tex}
% If you want glossary uncomment below -- comment out the two lines above
%\printglossaries


\end{document}
